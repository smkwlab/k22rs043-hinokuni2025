%%
%% 研究報告用スイッチ
%% [techrep]
%%
%% 欧文表記無しのスイッチ(etitle,eabstractは任意)
%% [noauthor]
%%

%\documentclass[submit,techrep]{ipsj}
\documentclass[submit,techrep,noauthor]{ipsj}



\usepackage[dvips]{graphicx}
\usepackage{latexsym}

\usepackage{here}


\def\Underline{\setbox0\hbox\bgroup\let\\\endUnderline}
\def\endUnderline{\vphantom{y}\egroup\smash{\underline{\box0}}\\}
\def\|{\verb|}
%

%\setcounter{巻数}{59}%vol59=2018
%\setcounter{号数}{10}
%\setcounter{page}{1}


\begin{document}


\title{演習支援システム「サポちゃん」における\\
管理者機能の開発と運用負担の軽減
}

\etitle{Development of Administrator Functions and Reduction of \\ Operational Burden in the Exercise Support System “Sapo-chan”}

\affiliate{KyusyuSangyoUniversityRS}{九州産業大学理工学部情報科学科\\
Department of Information Science, Faculty of Science and Technology, Kyusyu Sangyo University, Fukuoka 813--8503, Japan}

\affiliate{FukuokaWomensUniversity}{福岡女子大学国際文理学部環境科学科\\
Department of Environmental Sciences, Faculty of International Arts and Sciences,  Fukuoka Women's University, Fukuoka 813--8529, Japan}


\author{川口 諒}{Kawaguchi Ryo}{KyusyuSangyoUniversityRS}
\author{神屋 郁子}{Kamiya Yuuko}{FukuokaWomensUniversity}
\author{下川 俊彦}{Shimokawa Toshihiko}{KyusyuSangyoUniversityRS}

\begin{abstract}
九州産業大学理工学部のプログラミング系科目では,受講生がPCで演習に取り組み,SA/TA・教員が提出状況を確認する形式の授業が行われている.
我々の研究室ではこれを支援するシステム「サポちゃん」を開発してきたが,管理者向け機能が未整備で,データベースを直接操作する必要があり,運用負担やヒューマンエラーのリスクが課題であった.
さらに,クォーター制への未対応や授業情報の一括登録不可,SA/TAが過去のチェックコメントを確認できない点などの問題もあった.
そこで本研究では「サポちゃん2025」を開発し,管理者機能の拡充とこれらの運用上の課題を解決した.
評価アンケートの結果,高い有効性が確認された.
\end{abstract}

\begin{eabstract}
  In programming courses at Kyushu Sangyo University's Faculty of Science and Engineering, students complete exercises on PCs while SAs/TAs and instructors monitor submission status.
  Our laboratory developed the “Sapo-chan” system to support this process, but its lack of administrator functions required direct database manipulation, posing operational burdens and risks of human error. 
  Furthermore, issues included lack of support for the quarter system, inability to batch-register course information, and SAs/TAs being unable to review past check comments. 
  Therefore, this research developed “Sapo-chan 2025,” enhancing administrator functions and resolving these operational challenges. Evaluation questionnaires confirmed its high effectiveness.
\end{eabstract}
%
%\begin{jkeyword}
%情報処理学会論文誌ジャーナル,\LaTeX,スタイルファイル,べからず集
%\end{jkeyword}
%
%\begin{eabstract}
%This document is a guide to prepare a draft for submitting to IPSJ
%Journal, and the final camera-ready manuscript of a paper to appear in
%IPSJ Journal, using {\LaTeX} and special style files.  Since this
%document itself is produced with the style files, it will help you to
%refer its source file which is distributed with the style files.
%\end{eabstract}
%
%\begin{ekeyword}
%IPSJ Journal, \LaTeX, style files, ``Dos and Dont's'' list
%\end{ekeyword}

\maketitle

%1
\section{はじめに}
九州産業大学理工学部のプログラミング系科目では,受講生がPCを用いて演習課題に取り組み,SA/TA・教員が提出状況や正誤を確認する形式の授業が行われている.
これらの授業では,受講生の理解を促進するため,質問しやすい環境の整備と,SA/TA・教員による演習チェックの効率化が求められている.
質問や演習チェックが円滑に行えない場合,受講生の理解不足や学習時間の損失につながる.

我々の研究室では,これらを支援することを目的として,演習支援システム「サポちゃん」を開発している.
サポちゃんは,質問や演習チェックの依頼をボタン操作で行える機能を備え,授業運営の効率化に寄与している.

一方で,これまでの開発は受講生およびSA/TA・教員向け機能が中心であり,管理者向け機能が十分に整備されていない.
そのため,授業情報の登録にはデータベースの直接操作が必要となり,専門的知識を要するうえ,作業負担やヒューマンエラーのリスクが高い.
また,授業情報を事前に一括登録できないという課題もある.
さらに,令和7年度から導入されたクォーター制に対し,サポちゃんはセメスター制を前提としており,制度変更に対応していない.

加えて,SA/TA・教員用画面では,過去のチェックコメントを確認できず,再提出時にも全体確認が必要となるほか,未提出と再提出要求の状態を区別できないため,演習状況を把握しにくいという問題がある.

本研究の目的は,サポちゃんにおける管理者の運用負担を軽減し,クォーター制に対応させるとともに,SA/TA・教員用画面の課題を解決することである.


%2
\section{演習支援システム「サポちゃん」}
サポちゃんとは,我々の研究室で開発している授業中の演習を支援するシステムである.
ここで演習とは,授業前または授業中に取り組み,提出する問題のことを指す.
対象となる授業は,受講生がPCを用いて演習に取り組む形式の授業である.

サポちゃんの画面は,受講生用画面とSA/TA・教員用画面に分かれている.
受講生は授業開始時に自身の座席番号をサポちゃんに登録し,その後サポちゃんを通して演習のチェック依頼や質問依頼を行う.
SA/TA・教員は,受講生からの依頼に対してチェック結果を登録したり,座席を確認して質問の対応に行くことができる.
また,SA/TA・教員は各座席の演習状況を一覧で確認できるため,効率的に演習のサポートが可能である.

%2.1
\subsection{サポちゃんの問題点と解決策}\label{sec:problems}
既存のサポちゃんには以下のような問題点がある.
\begin{itemize}
  \item 管理者の負担が大きい
  \item クォーター制に対応していない
  \item 事前に授業情報を登録できない
  \item 過去のチェックコメントを確認できない
  \item 未提出なのか再提出中なのかを判別できない
\end{itemize}
以下で各問題点とその解決策について述べる.

\subsubsection{管理者の負担が大きい}
1つ目は,システム管理者の負担が大きい点である.
既存のサポちゃんには管理者向けの機能が十分に整備されていない.
そのため,授業情報を登録・編集・削除する際にデータベースを直接操作する必要がある.
ここで授業情報とは管理者がサポちゃんのデータベースに登録するユーザー・科目・授業・演習などの情報を指す.
このような運用では,データベースに関する基本的な知識や,サポちゃんのデータベース構造への理解を有する者でなければ管理作業を行うことができない.
また,授業情報の登録には正確性が求められる.
特に初回授業時には,登録すべき情報量が多く,複数のテーブルへの入力が必要となるため,作業が煩雑で手間がかかる.
さらに,この管理方法だと入力ミスや削除ミスなどのヒューマンエラーが発生するリスクも高いと考える.

本研究では,この問題点を解決するために,管理者機能を新たに開発し,管理者の負担を軽減する.
管理作業をWebのGUI上で完結させることでデータベースの直接操作を不要にする.
また,煩雑であった管理作業をわかりやすいGUIで実施可能にすることで,ヒューマンエラーの発生を防止する.
これにより,データベースに関する知識を持たない者でも管理可能なシステムを実現する.

\subsubsection{クォーター制に対応していない}

2つ目は,クォーター制に対応していない点である.
九州産業大学では令和7年度からクォーター制が導入され,1つの科目が週に複数回開講されるケースも多くなった.
しかし,既存のサポちゃんはセメスター制を前提として設計されており,授業回数や時間割の構造がクォーター制に適合していない.

そのため,週2回開講される科目を1つの科目として管理できない.
例えば,本来は同一科目として扱われるべき「月曜4限の授業」と「木曜4限の授業」であっても,サポちゃんでは別々の科目として登録する必要がある.

本研究では,この問題点を解決するために週2回以上開講される科目でも1つの科目として管理できるシステムに改良する.

\subsubsection{事前に授業情報を登録できない}
3つ目は,授業情報を事前にまとめて登録できない点である.
既存のサポちゃんでは,受講生側・SA/TA・教員側の画面ともに,データベースに格納されている最新の授業情報を表示する仕様となっている.
そのため,例えば14回分の授業情報をあらかじめ登録してしまうと,14回目の授業情報しか表示されない.
結果として,管理者は毎週の授業前に授業情報を登録する必要があり,運用の負担となっている.

本研究では,この問題点を解決するため,事前に授業情報を登録できるシステムに改良する.
また,日付を判別して該当する授業回の情報を表示する機能を追加する.

\subsubsection{過去のチェックコメントを確認できない}
4つ目は,SA/TA・教員が過去に登録したチェックコメントを確認できない点である.
既存のサポちゃんでは,再提出された演習に対して過去にどのようなコメントが付けられたのかを確認できない.
そのため,これまでSA/TAは前回の指摘内容を把握できないまま,提出されたコードやファイル全体を再度確認する必要があり,演習チェックに手間がかかっていた.

本研究では,この問題点を解決するため,演習チェックを行う画面に過去のチェックコメントを表示する機能を追加する.

\subsubsection{未提出なのか再提出中なのかを判別できない}
5つ目は,各受講生が「一度も演習を提出していない状態」なのか「提出済みではあるが再提出が必要な状態」なのかを判別できない点である.
サポちゃんには各座席の演習状況を確認する機能があり,SA/TA・教員はチェックがOKとなった演習数を確認できる.
しかし,提出済みで再提出を求められている状態と,一度も提出されていない状態とが,画面上では同一の状態として扱われている.

その結果,SA/TA・教員が受講生の演習状況を正確に把握できず,指導効率の低下を招いている.
そこで本研究では,未提出と再提出の状態を明確に判別できるよう,表示方法を改良する.

\section{サポちゃん2025の設計}
本研究で開発するシステムを「サポちゃん2025」と命名する.
サポちゃん2025における要件定義,機能一覧,について以下で述べる.

\subsection{要件定義}\label{sec:youken}
\ref{sec:problems}で述べた問題点を解決するために,サポちゃん2025で追加する要件を以下に示す.
ここで,ユーザーとはサポちゃんを利用する教員および学生を指す.
 科目とはシラバスに登録されるもので,複数回の授業で構成されるものを指す.
 授業とは時間割の一コマ分の時間のことを指す.

\begin{itemize}
    \item システム全体
    \begin{itemize}
      \item クォーター制への対応
    \end{itemize}
    \item 管理者
    \begin{itemize}
        \item ユーザー管理
        \begin{itemize}
            \item ユーザーの新規登録
            \item ユーザーの編集
            \item ユーザーの閲覧
            \item ユーザーの削除
        \end{itemize}
        \item 科目管理
        \begin{itemize}
            \item 科目の新規登録
            \item 科目の編集
            \item 科目の閲覧
            \item 科目の削除
            \item 担当SA/TAの割り当て
            \item 担当SA/TAの閲覧
            \item 担当SA/TAの解除
            \item 受講生の割り当て
            \item 受講生の閲覧
            \item 受講生の解除
        \end{itemize}
        \item 授業管理
        \begin{itemize}
            \item 授業の新規登録(事前の一括登録にも対応)
            \item 授業の編集
            \item 授業の閲覧
            \item 授業の削除
        \end{itemize}
    \end{itemize}

    \item SA/TA・教員
    \begin{itemize}
      \item チェックコメント履歴の閲覧
      \item 演習提出状況の可視化
    \end{itemize}

  \end{itemize}

\section{機能一覧}\label{sec:kinouichiran}

\ref{sec:youken}で定義した要件を満たすために,サポちゃん2025で新たに追加する機能の概要を以下に示す.

\subsection{管理者:ユーザー管理}
\begin{description}
  \item[ユーザー登録機能]
    ユーザー(教員・学生)を個別に登録する機能である.
    ユーザーID,パスワード,ユーザー名,ユーザー名(カナ),ユーザータイプ,管理者権限の有無を入力することで登録できる.
    なお,九州産業大学の統合認証システムに登録されているユーザーはLDAP認証を利用できるため,パスワード欄を空欄のまま登録可能である.
  \item[ユーザー編集機能]
    登録されているユーザー情報を編集する機能である.
    ユーザーID,パスワード,氏名,ユーザータイプ,管理者権限の有無を更新できる.
    LDAP認証ユーザーについては,パスワード欄が空欄の状態でも更新が可能である.
  \item[ユーザー削除機能]
    指定したユーザーをシステムから削除する機能である.
  \item[ユーザー一覧表示機能]
    各ユーザーのユーザーID,氏名,ユーザータイプ,管理者権限の有無を表示する機能である.
    20名ごとにページネーションで切り替えて表示する.
  \item[ユーザー検索機能]
    ユーザーID,ユーザー名,ユーザー名(カナ)を用いた部分一致検索を行う機能である.
  \item[教員登録機能]
    CSVまたはExcelファイルを用いて教員を一括登録する機能である.
    既に登録済みのユーザーは自動的にスキップされる.
  \item[受講生割り当て機能]
    指定した科目に受講生を割り当てる機能である.
    検索した既存ユーザーを割り当てる「個別割り当て」と,K's Lifeから取得した受講生リスト(Excel)を用いる「一括割り当て」が可能である.
    未登録の学生が含まれる場合は,自動的にユーザー登録も行われる.
  \item[受講生一覧表示機能]
    科目ごとの受講生を一覧表示する機能である.
    ユーザーID,氏名を表示し,20名ごとにページネーションで切り替えて表示する.
  \item[受講生解除機能]
    科目に対する受講生の割り当てを解除する機能である.
  \item[担当SA/TA割り当て機能]
    科目にSA/TAを割り当てる機能である.
    「個別割り当て」と,CSV/Excelファイルによる「一括割り当て」に対応している.
  \item[担当SA/TA一覧表示機能]
    科目に割り当てられているSA/TAの情報を一覧で確認できる機能である.
  \item[担当SA/TA解除機能]
    科目に割り当てられているSA/TAの権限を解除する機能である.
\end{description}

\subsection{管理者:科目管理}
\begin{description}
  \item[科目登録機能]
    科目を登録する機能である.
    手動で各項目(科目名,教員,年度,教室,曜日・時限・クォーターの2組)を入力する「個別登録」と,CSV/Excelを用いる「一括登録」を選択できる.
  \item[科目一覧表示機能]
    科目名,教員名,教室,年度,開講時間(曜日と時限の組み合わせ),クォーターを表示する.
  \item[科目編集機能]
    登録済み科目の基本情報や開講枠の情報を編集する.
  \item[科目削除機能]
    指定した科目を削除する.
\end{description}

\subsection{管理者:授業管理}
\begin{description}
  \item[授業登録機能]
    科目に紐づく各回の授業を登録する.「個別登録」のほか,15回分などをまとめて登録できる「一括登録」を備える.
    演習項目が複数ある場合はコンマ区切りで登録可能である.
  \item[授業一覧表示機能]
    授業回,日付,内容,演習項目を一覧で表示する.
  \item[授業編集機能]
    登録されている授業の詳細情報を更新する.
  \item[授業削除機能]
    授業を削除する機能である.
\end{description}

\subsection{SA/TA・教員用機能}
\begin{description}
  \item[チェックコメント一覧表示機能]
    演習チェック画面において,過去のコメント履歴を表示する機能である.
    過去の指導内容を確認しながら効率的にチェックを行える.
  \item[提出状況表示機能]
    受講生の演習提出状況を座席表の色で識別する機能である.
    未提出と「少なくとも一度提出がある状態」を視覚的に分けることで,進捗を直感的に把握できる.
\end{description}

\subsection{全ユーザー共通機能}
\begin{description}
  \item[時間割表クォーター切り替え機能]
    時間割表の表示対象(1Q~4Q)をボタンで切り替える機能である.
    ユーザー自身の属性(教員,SA/TA,受講生)に応じた科目のみが表示される.
\end{description}


\section{評価}
以下で評価方法,評価結果,考察について述べる.

\subsection{評価方法}
本研究では,評価の目的に応じて対象者を4つのグループに分類し,それぞれに評価対象となる機能を実際に使用してもらった上で,アンケート調査による評価実験を行った.
評価内容と対象者を表\ref{tb:hyouka}に示す.
また,アンケートの回答は5段階評価と自由記述を組み合わせたものであり,5段階の選択肢は以下の通りである.
\begin{enumerate}
  \item 全くそう思わない
  \item あまりそう思わない
  \item どちらともいえない
  \item 少しそう思う
  \item とてもそう思う
\end{enumerate}

\begin{table}[H]
  \centering
  \caption{評価内容と対象者}
  \begin{tabular}{|p{3cm}|p{5cm}|}\hline
    \multicolumn{1}{|c|}{評価内容} & \multicolumn{1}{c|}{対象者}  \\ \hline\hline
    管理者機能の評価
      & 過去にサポちゃんのデータベースを操作した経験のある者
       \\ \hline
    SA/TA用機能の改善点の評価
      & 過去にSA/TAとしてサポちゃんを利用した経験のある者
       \\ \hline
    システム全体の評価
      & 教員
       \\ \hline
    受講生用機能の評価
      & サポちゃんを使った経験がない受講生
       \\ \hline
  \end{tabular}
  \label{tb:hyouka}
\end{table}

\subsection{評価結果}
管理者機能の評価については,従来の管理作業と比較した負担軽減の観点からアンケートを実施した.

また,SA/TA用の機能の評価については,演習チェック作業の負担軽減や各受講生における演習状況の把握のしやすさといった観点からアンケートを実施した.

さらに,システム全体についての評価として,画面構成の分かりやすさや操作性などの観点から,教員を対象としたアンケートを実施した.


加えて,4Q「プログラミング基礎II」の12回目授業において,サポちゃんを使用した経験のない受講生を対象に受講生用の機能を利用してもらった.
これまでTeamsで行っていた演習チェックや挙手での質問方法と比較した利便性などの観点からアンケートを実施した.


\section{評価結果}\label{sec:hyoukaresult}
以下に各アンケートの評価結果を示す.
自由記述の回答は一部を要約して掲載する.
なお,括弧内の数字は,各回答者が選択した選択肢を示している.
また,一部の回答の明らかな誤字・脱字は修正して掲載している.

\subsection{管理者機能の評価結果}
3名から回答があった.
管理者機能の評価アンケートの結果を表\ref{tb:kanrisya}に示す.

\begin{table}[H]
  \centering
  \caption{管理者機能の評価結果}
  \begin{tabular}{|p{7cm}|c|}\hline
    \multicolumn{1}{|c|}{アンケート項目} & \multicolumn{1}{c|}{平均}  \\ \hline\hline
    Q1.科目・授業・演習管理画面を使用することで,科目・授業・演習管理の負担が軽減された.
      &5.0
      \\ \hline
    Q3.授業情報を事前に登録できるようになったことで,授業登録の負担は軽減された.
      &5.0
      \\ \hline
    Q5.ユーザー管理画面を使用することで,ユーザー管理の負担は軽減された.
      &5.0
       \\ \hline
    Q7.教員,受講生,SA/TAや科目,授業・演習を一括で登録できるようになったことで,登録作業の負担は軽減された.
      &5.0
       \\ \hline
    Q9.管理者機能を開発したことで,全体的に管理の負担は軽減された.
      &5.0 
       \\ \hline
  \end{tabular}
  \label{tb:kanrisya}
\end{table}

\begin{itemize}
    \item 「Q2.Q1の解答について,理由を教えてください」への回答
    \begin{itemize}
        \item 一覧で表示されることによって,目的のものを探す手間が減ったため.(5)
        \item UIもわかりやすく演習追加など簡単にできる.(5)
    \end{itemize}
    \item 「Q4.Q3の解答について,理由を教えてください」への回答
    \begin{itemize}
        \item 以前は授業前に毎回登録しないといけなかったため.(5)
        \item 授業の情報がわかった時点で一気に登録できるのが楽(5)
    \end{itemize}
    \item 「Q6.Q5の解答について,理由を教えてください」への回答
    \begin{itemize}
        \item SA/TAの追加やメンバーの削除時などにDBを直接見て編集する必要がなくなったから.(5)
        \item SA/TAを登録する際に,学籍番号や氏名で検索でき,視覚的にわかりやすくて楽(5)
    \end{itemize}
    \item 「Q8.Q7の解答について,理由を教えてください」への回答
    \begin{itemize}
        \item あらかじめある受講生の名簿ファイルをインポートするだけで登録できるのでとても楽(5)
        \item 今までphpMyAdminから手動で登録していたのが,インポートできるようになり非常に簡単になっているため.(5)
    \end{itemize}
    \item 「Q10.Q9の解答について,理由を教えてください」への回答
    \begin{itemize}
        \item DBを直接操作する必要がなくなり,登録する際のミスが大幅に減ると感じたため.(5)
        \item 今までDBを直接操作して面倒だった作業が全て改善されているので,負担はかなり軽減されていると思うから.(5)
    \end{itemize}
    \item 「Q11.その他ご意見などがあればお願いします.」への回答
    \begin{itemize}
        \item たまにスマホで見たりするので,スマホのUIもよかったら嬉しい
        \item スマホ版のUIが可能であればほしい.
    \end{itemize}
\end{itemize}

\subsection{SA/TA・教員用画面の改善内容の評価結果}
8名から回答があった.
SA/TA・教員用画面の改善内容の評価アンケートの結果を表\ref{tb:sata}に示す.

\begin{table}[H]
  \centering
  \caption{SA/TA・教員用画面の改善内容の評価結果}
  \begin{tabular}{|p{7cm}|c|}\hline
    \multicolumn{1}{|c|}{アンケート項目} & \multicolumn{1}{c|}{平均}  \\ \hline\hline
    Q1.過去のチェックコメントを確認できるようになったことで,演習チェックの負担が軽減された.
      &5.0
      \\ \hline
    Q3.演習の未提出と提出有を区別して表示することで,受講生の演習状況を把握しやすくなった.
      &4.9
      \\ \hline
  \end{tabular}
  \label{tb:sata}
\end{table}


\begin{itemize}
    \item 「Q2.Q1の解答について,理由を教えてください」への回答
    \begin{itemize}
        \item 他の人がNG出してた人をチェックする際になぜNGなのかがわからなかったため全体をチェックしないといけなかったが,NG出ている部分だけを見ることができるようになったため(5)
        \item 過去のチェックコメントみれるようになったことで指摘箇所だけ確認すればよくなったため,確認作業が簡単になったから.(5)
    \end{itemize}
    \item 「Q4.Q3の解答について,理由を教えてください」への回答
    \begin{itemize}
        \item 演習が未提出の学生に声をかけやすく,サポートしやすくなった.(5)
        \item 未提出で悩んでいるのか,再提出で修正作業をしているかの判別がつきやすくなり,サポートに行きやすくなったから.(5)
    \end{itemize}
\end{itemize}

\subsection{システム全体の評価結果}
3名から回答があった.
システム全体の評価アンケートの結果を表\ref{tb:zentai}に示す.
\begin{table}[H]
  \centering
  \caption{システム全体の評価結果}
  \begin{tabular}{|p{5cm}|c|}\hline
    \multicolumn{1}{|c|}{アンケート項目} & \multicolumn{1}{c|}{平均}  \\ \hline\hline
    Q1.画面の構成は分かりやすい.
      &4.7
      \\ \hline
    Q3.UIは直感的で操作しやすい.
      &5.0
      \\ \hline
    Q5.今後の授業でサポちゃんを使ってみたい.
      &4.0
      \\ \hline
  \end{tabular}
  \label{tb:zentai}
\end{table}

以下に自由記述の回答をすべて掲載する.
なお,括弧内の数字は,各回答者が選択した選択肢を示している.

\begin{itemize}
    \item 「Q2.Q1の解答について,理由を教えてください」への回答
    \begin{itemize}
        \item 必要な情報が表示されていたため.(5)
        \item 全体的に操作感として迷うような表記はなかったので良いと思う.
    \end{itemize}
    \item 「Q4.Q3の解答について,理由を教えてください」への回答
    \begin{itemize}
        \item 特に操作に困らなかったため.(5)
        \item 登録手順やボタンの名前などは直感的で迷うことはなかった.(5)
    \end{itemize}
    \item 「Q6.Q5の解答について,理由を教えてください」への回答
    \begin{itemize}
        \item 作業が簡単になったとしても作業自体は存在するので,それを上回るメリットがあれば検討したい.(3)
        \item 教員側は誰が演習がまだ終わっていないのかが視覚的に確認しやすいので良いと思う.
        また,手を挙げて質問しずらい子の質問にも気づきやすくなるので,学生側の評判も確認しながら細かい部分で機能が改善されるようであれば今後の授業でも使ってみたい.(4)
        \item 便利だから.(5)
    \end{itemize}
    \item 「Q7.その他ご意見などがあればお願いします.」への回答
    \begin{itemize}
        \item MoodleあるいはTeamsとの連携で登録作業が不要になり,学生からも操作がシームレスになればよいと思う.
        \item 履修登録名簿をインポートするだけで登録が完了するなど管理者側の使用感は格段に使いやすいと思うので,全く知らないツールとして使う障壁はだいぶ低いと思う.
        \item サポちゃんを使う管理者権限者(教員)が複数になってくると,他教員の授業の内容を勝手に変更や修正ができてしまうので,システム全てを編集できる権限ユーザー(管理者)と自分で登録した授業内容だけを編集できる権限ユーザー(教員),で分ける方が良いと思う.
        
    \end{itemize}
\end{itemize}

\subsection{受講生用画面の評価結果}
20名から回答があった.また,うち9名がサポちゃんを使って質問したと回答した.
受講生用画面の評価アンケートの結果を表\ref{tb:zyukousei}に示す.
\begin{table}[H]
  \centering
  \caption{受講生用画面の評価結果}
  \begin{tabular}{|p{7cm}|c|}\hline
    \multicolumn{1}{|c|}{アンケート項目} & \multicolumn{1}{c|}{平均}  \\ \hline\hline
    Q1.演習チェック依頼はTeamsと比べてしやすくなった.
      &4.7
      \\ \hline
    Q3.サポちゃんを使うことで質問はしやすくなった.(サポちゃんで質問した人のみ)
      &4.7
      \\ \hline
    Q5.今後の授業ではTeamsよりもサポちゃんを利用したい.
      &4.5
      \\ \hline
  \end{tabular}
  \label{tb:zyukousei}
\end{table}

\begin{itemize}
    \item 「Q2.Q1の解答について,理由を教えてください.」への回答
    \begin{itemize}
        \item 今待っている人数などがわかるのがよかった.また,クリック一つでできるのが楽だった.(4)
        \item Teamsでは,学籍番号と名前,「完成しました.」という文を打ちこまなければならなかったので,楽になったと感じました.(5)
    \end{itemize}
    \item 「Q4.Q3の解答について,理由を教えてください.」への回答
    \begin{itemize}
        \item 手を上げにくい人用に質問のボタンがあって気軽に質問しやすくなったと思います(4)
        \item 挙手だったら気付いてもらえないときがあるから(5)
    \end{itemize}
    \item 「Q6.Q5の解答について,理由を教えてください.」への回答
    \begin{itemize}
        \item いろいろな手間が省けていいと思うから.(5)
        \item 手を挙げても気づかれなかったり,先生方も手を挙げてないか見てないといけないといけなかったと思うので,便利になったと思いました.(5)
    \end{itemize}
    \item 「Q7.その他ご意見などがあればお願いします.」への回答
    \begin{itemize}
        \item とても便利で使いやすかったです.
    \end{itemize}
\end{itemize}
\section{考察}\label{sec:kousatsu}
評価アンケート結果について,以下に考察を述べる.

管理者機能の評価結果については,全ての設問で平均5.0という非常に高い評価を得ることができた.
自由記述の回答内容からも,管理者機能の開発により管理作業の負担が軽減できたことが分かる.
ただし,スマートフォンでの利用に関する要望があった.
管理者機能は主にPCで利用されることを想定していたが,スマートフォンでの利用も考慮したUI設計を今後検討する必要がある.

SA/TA・教員用画面の改善内容の評価結果については,どちらの設問も平均4.9以上という高い評価を得ることができた.
自由記述の回答内容からも,演習チェック作業の負担軽減や受講生の演習状況の把握のしやすさが向上したことが分かる.
特に過去のチェックコメントの確認機能については回答者全員が「とてもそう思う」と回答しており,重要な改善点であったことが分かる.

システム全体の評価結果については,全ての設問で平均4.0以上という高い評価を得ることができた.
自由記述の回答内容からも,画面構成の分かりやすさや操作性について肯定的な意見が多く寄せられた.
一方で,管理者権限の細分化や表示の工夫に関する要望もあった.
管理者権限の細分化についてはヒューマンエラー防止に関わるものであり,改善することでシステムの信頼性向上につながると考える.
また,座席図の上下反転機能が欲しいという教員目線の要望もあったため,今後のシステム改善において検討する必要がある.

受講生用画面の評価結果については,全ての設問で平均4.5以上という高い評価を得ることができた.
自由記述の回答内容からも,演習チェック依頼や質問依頼のしやすさはTeamsを用いた方法と比較してもサポちゃんの方が便利であると言える.
特に演習チェック依頼のしやすさについては,多くの回答者が「とてもそう思う」と回答しており,サポちゃんは受講生にとって利便性の高いシステムであることが分かる.

評価結果より,管理者,SA/TA・教員の各利用者の負担を軽減できたと言える.
また,受講生にとっても利便性の高いシステムであることが示された.

ただし,いくつかの改善点や要望も明らかになったため,今後のシステム改善においてこれらの点を考慮する必要がある.



\end{document}
